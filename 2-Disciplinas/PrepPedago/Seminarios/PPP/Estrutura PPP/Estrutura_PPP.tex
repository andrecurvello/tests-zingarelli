\documentclass[a4paper]{article}

\usepackage[utf8]{inputenc}
\usepackage[brazil]{babel}
\usepackage{indentfirst}

\author{Matheus Ricardo Uihara Zingarelli - 5377855}
\title{Estrutura Organizacional de um \\Projeto Político Pedagógico (PPP)}

\begin{document}
\maketitle

\section{PPP - Bacharelado em Matemática UFSCar}

Dividido nas seguintes seções:
\begin{itemize}
\item Histórico do curso

Abrange um histórico dos cursos de Matemática oferecidos pela Universidade e como foi sua evolução com o tempo, da adequação de cada curso à criação de novos cursos ou horários, baseados na necessidade da Universidade, resoluções governamentais, pesquisas de qualidade e pedidos dos discentes.

\item Papel social e atuação profissional

Destaque da importância da Matemática para a humanidade e de seu papel na educação com o desenvolvimento da postura crítica e abstrata do aluno na resolução de problemas e interpretação de dados. Mostra também o papel interdisciplinar da Matemática com outras áreas através de exemplos bem como de avanços futuros e problemas ainda a serem abordados.

Esta seção também explica que o PPP foi desenvolvido com o compromisso principal de preparar o aluno para a pós-graduação na área ou em áreas correlatas.

\item Definição do profissional

São elencados os itens que eles consideram que o profissional irá desenvolver ao se formar no curso.

\item Competências, habilidades, atitudes e valores

São elencados diversos itens que eles consideram necessários para o exercício da profissão, sendo de natureza científica, técnica, sócio-política, filosófica, ética, psicológica e profissional.

\item Divisão do conhecimento em grupos e seleção do conteúdo pertencente a cada um

Para trabalhar as competências descritas no item anterior, é dividido o conhecimento em grupos, e para cada grupo é descrito o conteúdo abordado. O objetivo dessa divisão é para organização curricular.

\item Descrição de disciplinas correspondentes a cada grupo

Dos grupos obtidos no item anterior, são distribuídas as disciplinas compostas pelo curso, baseado no conteúdo abordado por cada grupo.

\item Metodologia (como atingir os objetivos traçados para o profissional)

São descritos 3 princípios que devem ser sempre considerados no decorrer do curso: o desenvolvimento de competências através do aprendizado, a coerência entre teoria e prática através de exposição do aluno, e a atividade de pesquisa, considerada essencial na formação do profissional. Descreve-se também os processos de ensino que podem ser seguidos pelo professor para uma boa exposição do conhecimento.

\item Avaliação da aprendizagem

Considera a avaliação parte integrante do processo de formação, com a função de diagnosticar e corrigir os rumos tomados tanto pelo discente quanto pelo docente. As avaliações são constituídas por provas aplicadas ao aluno, avaliações da instituição externas independentes, exposição de resultados de pesquisa e a defesa do TCC perante uma banca examinadora. Descreve-se também outras formas de avaliação que podem ser aplicadas.

\item Formas de articulação entre disciplinas/atividades curriculares

Demonstra-se a ligação entre as disciplinas e atividades agrupadas e o modo que elas se inter-relacionam, não sendo independentes.

\item Bibliografia

Documentos de referência utilizados na produção do documento.

\item Anexos

Nos anexos encontramos diversos documentos informativos, como a grade proposta, ementa de cada disciplina, infraestrutura da Universidade disponível para o curso, quadro de docentes e servidores envolvidos no instituto, as regras para obtenção do diploma, relação das disciplinas distribuídas por departamento, relação das disciplinas por curso, histórico de mudanças no PPP.
\end{itemize}

{\bf Observação:} nota-se no PPP a formação de um profissional independente e que muitas vezes trabalha sozinho visando reflexão e formulação de soluções para um problema, o que podemos observar ser muito presente em vários alunos e professores de matemática. Porém, visa-se também a formação de um profissional que saiba trabalhar em um grupo composto de profissionais de diferentes áreas, algo que observa-se muitas vezes deixado de lados por muitos professores.

\section{PPP - Bacharelado em Ciência de Computação ICMC - USP}

Dividido nas seguintes seções:

\begin{itemize}

\item Objetivos do curso

Como o próprio nome diz, destaca as razões para o curso de Computação com base na capacitação profissional do aluno. Faz um resumo do que se espera do profissional a ser formado pelo curso, o qual é detalhado na seção seguinte.

\item Perfil do profissional a ser formado

Descreve as três áreas em que o profissional pode atuar e o conjunto de aptidões e habilidades consideradas necessárias serem exercidas tanto em todas as áreas como em cada área em específico.

\item Metodologia

Explicação da divisão das disciplinas pelos departamentos que a oferecem e como o conjunto de aptidões/habilidades é desenvolvido no decorrer do curso. Explica a divisão das disciplinas entre obrigatórias, visando formação básica na área de Computação e Informática, e optativas, visando formação específica para especialização em uma ou mais áreas da Computação.

Descreve também os métodos que podem ser utilizados para exposição do conhecimento pelos professores, bem como das atividades extracurriculares disponíveis para o aluno. 

As disciplinas, ênfases e atividades são então distribuídas conforme o desenvolvimento de aptidão/habilidade a ser adquirida, com uma pequena descrição do que é contemplado por cada.

\item Avaliação

O titulo da seção não é completo. Nesta seção é descrito a composição do curso entre professores e comissões, a relação candidato/vaga pelo vestibular e breve comentário sobre número de ingressantes, formados e desistentes.

Explica-se então as formas de avaliação do curso através dos alunos e destes através de estágio ou projeto de graduação. Não é falado sobre o método de avaliação do aluno em cada disciplina.

\item Programa de apoio aos alunos

Descreve alguns meios e programas que visam orientar e ouvir as necessidades dos alunos.

\item Apêndices

Apresenta 2 documentos, descrevendo a grade curricular proposta

\end{itemize}

\section{Aspectos comuns dos PPPs apresentados}

Ambos definem muito bem o perfil do aluno que eles esperam formar, tanto social quanto profissionalmente, e para qual direcionamento está voltado a estrutura do curso (acadêmico, empresarial, mesclado). Também elencam os vários objetivos do curso, com as disciplinas e atividades extracurriculares que servem como apoio para atingir tais objetivos.


\end{document}